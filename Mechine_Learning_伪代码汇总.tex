\documentclass{article}
\usepackage{amsmath} % 用于数学格式化
\usepackage[linesnumbered,ruled,vlined]{algorithm2e} % 使用algorithm2e包
\usepackage{ctex} % 用于中文字符支持
\usepackage{amssymb}

\title{机器学习算法伪代码记录}
\author{中南大学 Junyi Fang}
\date{2025年1月}

\begin{document}

\maketitle

\section{引言}
本文将记录一些常见的机器学习算法的核心思想和伪代码,以便在上课和上机实验过程中进行参考和回顾。注:其中梯度提升算法和弹性网络算法虽然上课没讲到,但是在上机中需要对算法进行调参评优,故自主学习记录。

\section{目录}
\begin{itemize}
    \item 梯度下降算法
    \item 逻辑回归算法
    \item KNN算法
    \item 神经网络(反向传播算法)
    \item K-means clustering 算法
    \item DBSCAN 算法
    \item 支持向量分类 (SVC)
    \item 决策树分类 (DecisionTreeClassifier)
    \item 随机森林分类 (RandomForestClassifier)
    \item 梯度提升分类 (XGBClassifier)*
    \item KNN分类 (KNeighborsClassifier)
    \item 多层感知器 (MLPClassifier)
    \item 朴素贝叶斯 (GaussianNB)
    \item AdaBoost 分类 (AdaBoostClassifier)
    \item 岭回归 (Ridge)
    \item Lasso 回归 (Lasso)
    \item 弹性网络 (ElasticNet)*
    \item 随机森林回归 (RandomForestRegressor)
    \item 梯度提升回归 (GradientBoostingRegressor)
    \item 自适应提升回归 (AdaBoostRegressor)
    \item 支持向量回归 (SVR)
    \item 决策树回归 (DecisionTreeRegressor)
\end{itemize}


\section{算法伪代码}

\begin{algorithm}
\caption{梯度下降算法(动量 + RMSprop)}
\KwIn{学习率 $\alpha$,动量系数 $\beta$,数值稳定性 $\epsilon$}
\textbf{初始化}:$\text{momentum} = 0$,$\text{squared\_grads} = 0$\;
\textbf{定义损失函数}:$L(x, y) = \sin(x) \cdot \cos(y)$\;
\textbf{定义梯度}:
\[
\frac{\partial L}{\partial x} = \cos(x) \cdot \cos(y), \quad \frac{\partial L}{\partial y} = -\sin(x) \cdot \sin(y)
\]
\textbf{训练过程}:
\For{每个训练步骤 $t = 1, 2, \ldots, T$} {
    计算梯度 $\nabla L = \left[ \frac{\partial L}{\partial x}, \frac{\partial L}{\partial y} \right]$\;
    更新动量:$\text{momentum} = \beta \cdot \text{momentum} + (1 - \beta) \cdot \nabla L$\;
    更新平方梯度:$\text{squared\_grads} = \beta \cdot \text{squared\_grads} + (1 - \beta) \cdot (\nabla L)^2$\;
    计算调整后的梯度:$\text{adjusted\_grads} = \frac{\text{momentum}}{\sqrt{\text{squared\_grads}} + \epsilon}$\;
    更新参数:$\text{params} = \text{params} - \alpha \cdot \text{adjusted\_grads}$\;
}
\textbf{输出}:返回最终的参数 $\text{params}$\;
\end{algorithm}

\begin{algorithm}
\caption{逻辑回归算法(通过梯度下降优化)}
\KwIn{学习率 $\alpha$,迭代次数 $T$,输入数据 $X$,标签 $y$}
\textbf{初始化}:$\text{weights} = \mathbf{0}$,$\text{bias} = 0$\;
\textbf{定义 Sigmoid 函数}:
\[
\sigma(z) = \frac{1}{1 + e^{-z}}
\]
\textbf{训练过程}:
\For{每个训练步骤 $t = 1, 2, \ldots, T$} {
    计算线性模型输出:$\text{linear\_model} = X \cdot \text{weights} + \text{bias}$\;
    计算预测值:$\text{predictions} = \sigma(\text{linear\_model})$\;
    计算梯度:$dw = \frac{1}{n} X^T (\text{predictions} - y)$,$db = \frac{1}{n} \sum (\text{predictions} - y)$\;
    更新权重:$\text{weights} = \text{weights} - \alpha \cdot dw$\;
    更新偏置:$\text{bias} = \text{bias} - \alpha \cdot db$\;
}
\textbf{输出}:返回最终的权重 $\text{weights}$ 和偏置 $\text{bias}$\;
\end{algorithm}

\begin{algorithm}
\caption{逻辑回归预测}
\KwIn{输入数据 $X$,权重 $\text{weights}$,偏置 $\text{bias}$}
计算线性模型输出:$\text{linear\_model} = X \cdot \text{weights} + \text{bias}$\;
计算预测概率:$\text{predicted\_probabilities} = \sigma(\text{linear\_model})$\;
根据预测概率判断类别:$\text{predicted\_class} = \{1 \text{ if } p > 0.5 \text{ else } 0 \, | \, p \in \text{predicted\_probabilities}\}$\;
\textbf{输出}:返回预测类别 $\text{predicted\_class}$\;
\end{algorithm}

\begin{algorithm}
\caption{K-Nearest Neighbors(KNN)算法}
\KwIn{输入数据 $X_{\text{train}}$,标签 $y_{\text{train}}$,测试数据 $X_{\text{test}}$,最近邻数目 $k$}
\textbf{初始化}:将训练数据 $X_{\text{train}}$ 和标签 $y_{\text{train}}$ 存储为模型的属性;
\For{每个测试样本 $x$ 在 $X_{\text{test}}$ 中} {
    计算每个训练样本与测试样本之间的距离:$\text{distances} = [\text{dis}(x, x_{\text{train}}) \text{ for each } x_{\text{train}} \text{ in } X_{\text{train}}]$\;
    找到最近的 $k$ 个邻居:$\text{k\_indices} = \text{argsort(distances)}[:k]$\;
    获取这些邻居的标签:$\text{k\_nearest\_labels} = [y_{\text{train}}[i] \text{ for } i \text{ in } k\_indices]$\;
    使用投票机制确定预测标签:$\text{most\_common} = \text{Counter}(k\_nearest\_labels).most\_common(1)$\;
    预测标签:$\text{predicted\_label} = \text{most\_common[0][0]}$\;
}
\textbf{输出}:返回每个测试样本的预测标签;
\end{algorithm}

\begin{algorithm}
\caption{神经网络(前向传播与反向传播)}
\KwIn{输入数据 $X$,标签 $y$,学习率 $\eta$}
\textbf{初始化}:
    权重:$\text{weights\_input\_hidden}$,$\text{weights\_hidden\_output}$,偏置:$\text{bias\_hidden}$,$\text{bias\_output}$;

\textbf{定义 Sigmoid 激活函数}:
\[
\sigma(x) = \frac{1}{1 + e^{-x}}
\]

\textbf{前向传播过程}:
\[
\text{hidden\_layer\_input} = X \cdot \text{weights\_input\_hidden} + \text{bias\_hidden}
\]
\[
\text{hidden\_layer\_output} = \sigma(\text{hidden\_layer\_input})
\]
\[
\text{output\_layer\_input} = \text{hidden\_layer\_output} \cdot \text{weights\_hidden\_output} + \text{bias\_output}
\]
\[
\text{output\_layer\_output} = \sigma(\text{output\_layer\_input})
\]

\textbf{反向传播过程}:
计算输出层误差:
\[
\text{d\_output} = (\text{output\_layer\_output} - y) \cdot \sigma'(\text{output\_layer\_output})
\]
计算隐藏层误差:
\[
\text{d\_hidden\_layer} = \text{d\_output} \cdot \text{weights\_hidden\_output}^T \cdot \sigma'(\text{hidden\_layer\_output})
\]

更新权重和偏置:
\[
\text{weights\_hidden\_output} -= \text{hidden\_layer\_output}^T \cdot \text{d\_output} \cdot \eta
\]
\[
\text{bias\_output} -= \sum \text{d\_output} \cdot \eta
\]
\[
\text{weights\_input\_hidden} -= X^T \cdot \text{d\_hidden\_layer} \cdot \eta
\]
\[
\text{bias\_hidden} -= \sum \text{d\_hidden\_layer} \cdot \eta
\]

\textbf{输出}:返回更新后的权重和偏置;
\end{algorithm}

\begin{algorithm}
\caption{K-means clustering 算法}
\KwIn{数据集 $data$,簇的数量 $num\_clusters$,最大迭代次数 $max\_iter$,收敛容忍度 $tol$}
\textbf{初始化}:随机选择 $num\_clusters$ 个数据点作为初始质心 $\text{centroids}$\;
\For{每次迭代 $i = 1, 2, \ldots, max\_iter$} {
    计算每个数据点到所有质心的距离:$\text{distances} = \text{norm}(data[:, None] - \text{centroids}, axis=2)$\;
    将每个数据点分配给最近的质心:$\text{labels} = \text{argmin}(\text{distances}, axis=1)$\;
    更新质心:$\text{new\_centroids} = \text{mean}(data[\text{labels} == i], axis=0) \text{ for each cluster $i$}$\;
    如果质心的变化小于容忍度,则停止迭代:$\text{if norm(new\_centroids - centroids)} \leq \text{tol} \text{ then break}$\;
    更新质心:$\text{centroids} = \text{new\_centroids}$\;
}
\textbf{输出}:返回最终的簇标签 $\text{labels}$ 和质心 $\text{centroids}$\;
\end{algorithm}

\begin{algorithm}
\caption{DBSCAN 算法}
\KwIn{数据集 $data$,距离阈值 $\epsilon$,最小邻居数 $min\_pts$}
\textbf{初始化}:标签 $\text{labels} = -1$,访问标记 $\text{visited} = 0$,簇标记 $\text{cluster\_id} = 0$\;
\textbf{邻居查找}:定义函数 $\text{neighbors}(point\_idx)$ 计算距离小于 $\epsilon$ 的点\;
\textbf{扩展簇}:定义函数 $\text{grow\_cluster}(idx, neighbors\_list)$ 来扩展簇,并对未访问的点进行标记\;
\For{每个数据点 $idx$} {
    如果已经访问过,则跳过;
    标记为已访问;
    查找邻居:$\text{neighbor\_pts} = \text{neighbors}(idx)$\;
    如果邻居数大于等于 $min\_pts$,则扩展簇:$\text{grow\_cluster}(idx, \text{neighbor\_pts})$\;
    增加簇标记 $cluster\_id$;
}
\textbf{输出}:返回每个点的簇标签 $\text{labels}$\;
\end{algorithm}
\begin{algorithm}
\caption{支持向量机分类 (SVC)}
\KwIn{训练数据 $X \in \mathbb{R}^{n \times p}$,标签 $y \in \mathbb{R}^{n}$,惩罚系数 $C \in \mathbb{R}$,核函数 $K$, 容忍度 $\epsilon$}
\textbf{初始化}:选择合适的核函数 $K$,设置惩罚参数 $C$,初始化拉格朗日乘子 $\alpha = 0$;
\For{每个训练样本 $i = 1, 2, \dots, n$} {
    计算核函数:$K(x_i, x_j)$;
    计算拉格朗日乘子 $\alpha_i$;
    通过最大化间隔,求解支持向量机优化问题:
    \[
    \min_{\mathbf{w}, b} \frac{1}{2} \|\mathbf{w}\|^2 \quad \text{s.t.} \quad y_i (\mathbf{w}^T x_i + b) \geq 1 - \xi_i, \quad \xi_i \geq 0
    \]
    其中 $\xi_i$ 是松弛变量,用于允许误差;
}
\textbf{输出}:返回支持向量 $\alpha$,权重向量 $\mathbf{w}$,偏置项 $b$;
\end{algorithm}

\begin{algorithm}
\caption{决策树分类 (Decision Tree Classifier)}
\KwIn{训练数据 $X \in \mathbb{R}^{n \times p}$,标签 $y \in \mathbb{R}^{n}$,最大深度 $D \in \mathbb{N}$,最小样本分割数 $min\_samples\_split \in \mathbb{N}$}
\textbf{初始化}:根节点为空;
\While{未满足停止条件} {
    选择最佳特征 $f$,并计算特征 $f$ 的划分点 $\theta$;
    计算信息增益或基尼指数:
    \[
    \text{Information Gain} = \text{Entropy}(S) - \sum_{v \in \text{Values}(f)} \frac{|S_v|}{|S|} \cdot \text{Entropy}(S_v)
    \]
    或者使用基尼指数:
    \[
    Gini = 1 - \sum_{i=1}^k p_i^2
    \]
    将数据集按特征 $f$ 和划分点 $\theta$ 划分为左右子集;
    递归构建左子树和右子树;
}
\textbf{输出}:返回构建的决策树;
\end{algorithm}

\begin{algorithm}
\caption{随机森林分类 (Random Forest Classifier)}
\KwIn{训练数据 $X \in \mathbb{R}^{n \times p}$,标签 $y \in \mathbb{R}^{n}$,决策树数量 $n\_trees \in \mathbb{N}$,最大深度 $D \in \mathbb{N}$}
\For{每棵树 $t = 1, 2, \dots, n\_trees$} {
    随机抽取数据集 $X_t \subset X$ 和 $y_t \subset y$,训练决策树 $t$;
    树的构建过程同决策树分类;
}
\textbf{输出}:通过所有树的投票机制,返回最终预测标签:
\[
\hat{y} = \arg\max_{y} \sum_{t=1}^{n\_trees} I(y_t = y)
\]
其中 $I$ 是指示函数,表示类别 $y$ 出现的次数;
\end{algorithm}

\begin{algorithm}
\caption{极限梯度提升*(XGBClassifier)}
\KwIn{训练数据 $X \in \mathbb{R}^{n \times p}$,标签 $y \in \mathbb{R}^{n}$,学习率 $\eta \in \mathbb{R}$,树的数量 $n\_estimators \in \mathbb{N}$}
\textbf{初始化}:初始模型为零函数 $f_0(x) = 0$;
\For{每棵树 $t = 1, 2, \dots, n\_estimators$} {
    计算负梯度:
    \[
    g_t = -\frac{\partial L}{\partial f_t(x)}
    \]
    用 $g_t$ 训练一棵回归树,获得树的预测 $h_t(x)$;
    更新模型:
    \[
    f_{t+1}(x) = f_t(x) + \eta \cdot h_t(x)
    \]
    其中 $L$ 是损失函数,通常为对数损失或平方误差;
}
\textbf{输出}:返回最终的模型 $f_T(x)$;
\end{algorithm}

\begin{algorithm}
\caption{K 近邻分类 (KNeighbors Classifier)}
\KwIn{训练数据 $X_{\text{train}} \in \mathbb{R}^{n_{\text{train}} \times p}$,标签 $y_{\text{train}} \in \mathbb{R}^{n_{\text{train}}}$,测试数据 $X_{\text{test}} \in \mathbb{R}^{n_{\text{test}} \times p}$,最近邻数量 $k \in \mathbb{N}$}
\For{每个测试样本 $x \in X_{\text{test}}$} {
    计算每个训练样本与测试样本之间的距离:
    \[
    \text{distances} = \|X_{\text{train}} - x\|
    \]
    找到最近的 $k$ 个邻居:
    \[
    \text{k\_indices} = \text{argsort}(\text{distances})[:k]
    \]
    预测标签:$\hat{y} = \text{majority\_vote}(y_{\text{train}}[\text{k\_indices}])$;
}
\textbf{输出}:返回每个测试样本的预测标签;
\end{algorithm}

\begin{algorithm}
\caption{多层感知器 (MLPClassifier)}
\KwIn{输入数据 $X \in \mathbb{R}^{n \times p}$,标签 $y \in \mathbb{R}^{n}$,学习率 $\eta \in \mathbb{R}$,隐藏层数量和神经元数量 $h_1, h_2, \dots, h_L$}
\textbf{初始化}:随机初始化权重 $W_{\text{input}}, W_{\text{hidden}}, \dots$ 和偏置 $b_{\text{input}}, b_{\text{hidden}}, \dots$;
\textbf{前向传播}:
\[
\text{hidden\_layer} = \sigma(X \cdot W_{\text{input}} + b_{\text{input}}), \quad \text{output\_layer} = \sigma(\text{hidden\_layer} \cdot W_{\text{hidden}} + b_{\text{hidden}})
\]
其中 $\sigma$ 是激活函数,例如ReLU或Sigmoid;
\textbf{反向传播}:
\[
\text{error\_output} = \text{output\_layer} - y, \quad \text{error\_hidden} = \text{error\_output} \cdot W_{\text{hidden}}^T \cdot \sigma'(\text{hidden\_layer})
\]
更新权重和偏置:
\[
W_{\text{hidden}} -= \eta \cdot \text{error\_output} \cdot \text{hidden\_layer}^T, \quad W_{\text{input}} -= \eta \cdot \text{error\_hidden} \cdot X^T
\]
\textbf{输出}:返回训练好的模型;
\end{algorithm}

\begin{algorithm}
\caption{朴素贝叶斯分类 (Gaussian Naive Bayes)}
\KwIn{训练数据 $X \in \mathbb{R}^{n \times p}$,标签 $y \in \mathbb{R}^{n}$}
\textbf{初始化}:根据训练数据 $X$ 和标签 $y$ 计算每个类别的先验概率:
\[
P(y = c) = \frac{\sum_{i=1}^{n} I(y_i = c)}{n}
\]
和每个特征的条件概率:
\[
P(x_j | y = c) = \frac{\sum_{i=1}^{n} I(y_i = c) \cdot x_{ij}}{\sum_{i=1}^{n} I(y_i = c)}
\]
\For{每个测试样本 $x$} {
    计算后验概率:
    \[
    P(y | X) = P(y) \prod_{j=1}^{p} P(x_j | y)
    \]
    选择最大后验概率的类别作为预测标签:
    \[
    \hat{y} = \arg\max_{c} P(y = c | X)
    \]
}
\textbf{输出}:返回每个测试样本的预测标签;
\end{algorithm}

\begin{algorithm}
\caption{AdaBoost 分类器 (AdaBoostClassifier)}
\KwIn{训练数据 $X \in \mathbb{R}^{n \times p}$,标签 $y \in \mathbb{R}^{n}$,弱分类器数量 $n\_estimators \in \mathbb{N}$}
\textbf{初始化}:样本权重 $w_i = \frac{1}{n}$;
\For{每个弱分类器 $t = 1, 2, \dots, n\_estimators$} {
    训练一个弱分类器 $h_t$;
    计算加权错误率:
    \[
    \text{error}_t = \frac{\sum_{i=1}^{n} w_i I(h_t(x_i) \neq y_i)}{\sum_{i=1}^{n} w_i}
    \]
    更新分类器权重:
    \[
    \alpha_t = \frac{1}{2} \ln \frac{1 - \text{error}_t}{\text{error}_t}
    \]
    更新样本权重:
    \[
    w_i = w_i \cdot e^{-\alpha_t y_i h_t(x_i)}
    \]
}
\textbf{输出}:返回最终加权分类器模型;
\end{algorithm}

\begin{algorithm}
\caption{岭回归 (Ridge Regression)}
\KwIn{训练数据 $X \in \mathbb{R}^{n \times p}$,标签 $y \in \mathbb{R}^{n}$,正则化系数 $\lambda \in \mathbb{R}$}
\textbf{初始化}:计算正规方程:
\[
\hat{\beta} = (X^T X + \lambda I)^{-1} X^T y
\]
\textbf{输出}:回归系数 $\hat{\beta} \in \mathbb{R}^{p}$
\end{algorithm}

\begin{algorithm}
\caption{Lasso 回归 (Lasso Regression)}
\KwIn{训练数据 $X \in \mathbb{R}^{n \times p}$,标签 $y \in \mathbb{R}^{n}$,正则化系数 $\lambda \in \mathbb{R}$}
\textbf{初始化}:回归系数 $\beta = 0$;
\For{每个特征 $j = 1, 2, \dots, p$}{
    计算坐标下降步长:$r_j = \frac{1}{n} \sum_{i=1}^{n} x_{ij} (y_i - \hat{y}_i + \beta_j x_{ij})$;
    更新回归系数:$\beta_j = \text{soft\_threshold}(r_j, \lambda)$
}
\textbf{输出}:回归系数 $\beta \in \mathbb{R}^{p}$;
\end{algorithm}

\begin{algorithm}
\caption{弹性网回归*(ElasticNet)}
\KwIn{训练数据 $X \in \mathbb{R}^{n \times p}$,标签 $y \in \mathbb{R}^{n}$,正则化系数 $\lambda_1, \lambda_2 \in \mathbb{R}$}
\textbf{初始化}:回归系数 $\beta = 0$;
\For{每个特征 $j = 1, 2, \dots, p$}{
    计算步长:$r_j = \frac{1}{n} \sum_{i=1}^{n} x_{ij} (y_i - \hat{y}_i + \beta_j x_{ij})$;
    更新回归系数:$\beta_j = \text{soft\_threshold}(r_j, \lambda_1 + \lambda_2)$;
}
\textbf{输出}:回归系数 $\beta \in \mathbb{R}^{p}$;
\end{algorithm}

\begin{algorithm}
\caption{随机森林回归 (Random Forest Regressor)}
\KwIn{训练数据 $X \in \mathbb{R}^{n \times p}$,标签 $y \in \mathbb{R}^{n}$,树的数量 $n\_trees \in \mathbb{N}$,最大深度 $D \in \mathbb{N}$}
\textbf{初始化}:随机森林模型 $\mathcal{F} = \{T_1, T_2, \dots, T_{n\_trees}\}$;
\For{每棵树 $t = 1, 2, \dots, n\_trees$}{
    随机选择数据子集 $X_t \subset X$,标签子集 $y_t \subset y$;
    构建回归树 $T_t$,最大深度为 $D$;
}
\textbf{输出}:预测值 $\hat{y} = \frac{1}{n\_trees} \sum_{t=1}^{n\_trees} T_t(X)$;
\end{algorithm}

\begin{algorithm}
\caption{梯度提升回归 (Gradient Boosting Regressor)}
\KwIn{训练数据 $X \in \mathbb{R}^{n \times p}$,标签 $y \in \mathbb{R}^{n}$,树的数量 $n\_estimators \in \mathbb{N}$,学习率 $\eta \in \mathbb{R}$}
\textbf{初始化}:初始模型 $\hat{y}_0 = \frac{1}{n} \sum_{i=1}^{n} y_i$;
\For{每棵树 $t = 1, 2, \dots, n\_estimators$}{
    计算残差:$\text{residuals}_t = y - \hat{y}_{t-1}$;
    训练回归树 $T_t$,拟合残差;
    更新模型:$\hat{y}_t = \hat{y}_{t-1} + \eta \cdot T_t(X)$;
}
\textbf{输出}:最终预测值 $\hat{y}_{final} = \hat{y}_{n\_estimators}$;
\end{algorithm}

\begin{algorithm}
\caption{自适应提升回归 (AdaBoost Regressor)}
\KwIn{训练数据 $X \in \mathbb{R}^{n \times p}$,标签 $y \in \mathbb{R}^{n}$,弱回归器数量 $n\_estimators \in \mathbb{N}$}
\textbf{初始化}:样本权重 $w_i = \frac{1}{n}$,初始预测模型为常数模型 $\hat{y}_0 = \frac{1}{n} \sum_{i=1}^{n} y_i$;
\For{每个弱回归器 $t = 1, 2, \dots, n\_estimators$}{
    训练回归器 $h_t$,并计算加权误差率:$\text{error}_t = \frac{\sum_{i=1}^{n} w_i \cdot (y_i - h_t(x_i))^2}{\sum_{i=1}^{n} w_i}$;
    计算回归器权重:$\alpha_t = \frac{1}{2} \ln \frac{1 - \text{error}_t}{\text{error}_t}$;
    更新样本权重:$w_i = w_i \cdot e^{-\alpha_t (y_i - h_t(x_i))^2}$;
}
\textbf{输出}:最终加权回归器模型 $\hat{y} = \sum_{t=1}^{n\_estimators} \alpha_t h_t(x)$;
\end{algorithm}

\begin{algorithm}
\caption{支持向量回归 (SVR)}
\KwIn{训练数据 $X \in \mathbb{R}^{n \times p}$,标签 $y \in \mathbb{R}^{n}$,惩罚系数 $C \in \mathbb{R}$,核函数 $K$,容忍度 $\epsilon \in \mathbb{R}$}
\textbf{初始化}:选择合适的核函数 $K$,设置惩罚参数 $C$,容忍度 $\epsilon$;
\For{每个训练样本 $i = 1, 2, \dots, n$}{
    求解拉格朗日乘子,最大化间隔,最小化以下目标函数:
    \[
    \mathcal{L}(\beta) = \frac{1}{2} \|\beta\|^2 + C \sum_{i=1}^{n} \max(0, |y_i - \hat{y}_i| - \epsilon)
    \]
}
\textbf{输出}:支持向量回归模型 $\hat{y} = f(X)$;
\end{algorithm}

\begin{algorithm}
\caption{决策树回归 (Decision Tree Regressor)}
\KwIn{训练数据 $X \in \mathbb{R}^{n \times p}$,标签 $y \in \mathbb{R}^{n}$,最大深度 $D \in \mathbb{N}$,最小样本分割数 $min\_samples\_split \in \mathbb{N}$}
\textbf{初始化}:根节点为空;
\While{未满足停止条件}{
    计算每个特征的最佳切分点:
    \[
    \text{BestSplit} = \arg\min_{\text{feature}, \text{threshold}} \text{MSE}_{\text{split}}
    \]
    将数据集按最佳切分点切分为左右子集;
    对每个子集递归调用决策树构建算法;
}
\textbf{输出}:决策树回归模型;
\end{algorithm}

\end{document}
